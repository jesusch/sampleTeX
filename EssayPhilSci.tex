\documentclass[12pt]{article}
\usepackage{csquotes}
\title{\textbf{Does the Dispositionalist still require the Laws of Nature in their Ontology?}}
\author{B080155}
\date{21.04.16}
\addtolength{\oddsidemargin}{-.875in}
\addtolength{\evensidemargin}{-.875in}
\addtolength{\textwidth}{1.3in}
\addtolength{\topmargin}{-1in}
\addtolength{\textheight}{1.75in}
\begin{document}
\maketitle

\section{Introduction}

In light of recent challenges from Mumford (2004), there has been debate amongst dispositionalists as to whether the laws of nature are still required in their ontology. 
This article shall investigate this issue whilst assuming dispositionalism to be the correct account of the laws of nature.\par
Dispositionalists claim that some (or all) properties are better understood as \textit{causal powers}. 
Said properties are those that predicate a disposition; such as `is soluble,' `is fragile,' or `is malleable.' For example when describing the properties of sugar, one may cite the property `is soluble.' 
A distinction should be drawn here between dispositional essentialists and dispositional monists. The former endorse a weaker view. Some assert that only some properties are dispositions, while others assert that every property has only a dispositonal element. 
The latter believe that all properties are dispositions. It should be noted that this article shall remain neutral with regards to the form of dispositionalism it subscribes to - provided that one accepts a dispositional account of the laws of nature then the wider debate surrounding dispositions should not be of concern for our present purposes.\par
Dispositionalists are subscribed to the essentialist notion of natural kinds. If one thought that an entity could hold contingent dispositions then one's view would be that of categoricalism as opposed to dispositionalism. 
A natural kind is typically thought of as a set of entities that share the necessary criteria for membership of their kind. These criteria are the essential properties of the kind. 
Putnam's classic example of a natural kind is that of water: Members of the natural kind, \textit{water}, can be identified by their satisfaction of the necessary condition of `being H2O.' Identifying a natural kind allows us to infer information about other members of said kind, as what can be said of one member can be said of all.\par
Dispositionalism aims to account for the laws of nature by appealing to the dispositional properties of natural kinds. 
This move allows for a robust account of the laws of nature wherein they become empirically grounded whilst still maintaining their governance over the world. 
This article shall explore Mumford's eliminativist arguments regarding the laws of nature in our ontology, and conclude that they are not easily dismissed. 
 
\section{The Laws of Nature}

The laws of nature have a controversial ontology. There are four main schools of thought on the laws of nature:\par

\begin{enumerate}
\item The Regularity Thesis: Also known as the Humean view. Laws are necessary but they are just placeholders for observable regularities in our world. 
\item Nomic Necessitation: This view is most notably held by Armstrong (1997). Laws contingently entail regularities in our world but do not necessitate them.
\item Dispositionalism: The properties of objects are causal powers. The laws of nature can be delineated from these powers and necessarily govern. 
\item Primitivism: The laws of nature fundamentally govern, but can be contingent or necessary.
\end{enumerate}
While the question of necessity remains somewhat controversial, there is a standard conception of the laws of nature that can be drawn from theses. 
The laws of nature must be necessary whilst playing a governing and explanatory role with regards to the states of affairs in our world. Dispositionalism is well placed to capitalise on these criteria.\par
Tugby illustrates how we can use the notion of dispositional essences to explain the phenomenon of charged particles accelerating in an electro-static field. 
He states: \begin{displayquote} In explaining a feature of this charge, dispositionalists take themselves to have said something about the essential nature of charge. (2013, 452.) \end{displayquote}
Dispositionalists account for the laws of nature by tethering them to the directedness of properties in the natural world. Central to the dispositionalist account of the laws of nature is the notion of causation. 
Dispositionalists hold that dispositions directly cause the laws of nature to exist by underpinning the phenomena that occurs in the world. In contrast to this, Humeans hold that we cannot ever give a robust account of causality. 
They think that at most we can say that events are merely constantly conjoint in patterns, as we can never know exactly what is causing events to take place. Dispositionalism allows us to say that laws are necessary because they are given truth-makers by the essential properties of a natural kind. 
A truth-maker is the corresponding state of affairs by virtue of which one can assert a proposition to be true or false. Any property that is essential is also \textit{necessary}, as the two predicates are co-extensive. \par 

Dispositionalism explains how the laws of nature are embedded into states of affairs in the world and how they govern said states of affairs. 
Supervenience is a metaphysical phenomenon wherein two sets of properties cannot independently alter their members without also altering the other set of properties. Their identities are essentially linked. 
By accounting for the laws of nature in terms of causal powers, dispositionalists are able to \textit{explain} how the laws supervene on to nature: If the properties of a charged particle were to alter then the laws of nature would also alter. 
The laws are given a \textit{governing} role in our ontology due to the properties of natural kinds abiding by their stipulations. 
If natural kinds are not following their expected behaviour then our understanding of the laws can organically alter to accommodate discrepancies without the laws themselves changing or becoming contingent.\par

To go back to Tugby's charged particles example, it is a law that charged particles accelerate in a static field because every member of the natural kind of charged particles is disposed towards accelerating in a static field.


\section{Mumford's Eliminativism}
\subsection{The Semantics of `Law'}

Mumford is a dispositionalist. Indeed he believes that adopting dispositionalism elegantly solves the problems associated with Humean accounts of laws by embedding a necessary metaphysic of laws into nature (2004, 120). 
However, he states that there are fundamental problems in holding a realist conception of laws. His central negative thesis is that the laws of nature are not clearly defined such that one can justify their existence: 
\begin{displayquote} Following Quine's famous dictum, no entity without identity, this gives us reason to doubt the existence of laws. There is \textit{such} vagueness in the concept of a law that it fails to refer. (Mumford, 2004, 127.)
 \end{displayquote} 
When we reflect on instances of a law in the sciences, we are confronted with a myriad of different examples. 
Mumford notes that we have some laws which are false if ``taken literally," and ``vacuous'' laws like Newton's First law of motion. 
He goes on to emphasise that even just within physics we have laws, principles, theorems, rules, equations and hypotheses all performing the same role (\textit{Ibid.}, 129-130). \par 
Mumford's first point concerns the content of the laws themselves. 
Laws can be highly idealised. Many Newtonian systems do not take full account of friction, and there are instances such as Boyle's Gas Law which relies heavily on idealisation in order to explain the relationship between volume and pressure in gas containers. 
Newton's first law is also an unusual candidate for a law due to its assertion that \textit{nothing} will occur. (Although it could perhaps be contested that it were ``vacuous'' given its foundational place in Newton's theory.) 
At any rate, it does not appear as though these laws are mapping onto \textit{real} entities. Mumford's second point targets the semantics of the word `law.' He is noting that there are a variety of terms that play the same functional role as what most would consider a law. He suggests that we have little reason to privilege the term `law' above others.\par
As an addendum to this point Mumford states that it is inconsistent for any metaphysical realist to hold that any false law is ``metaphysically real" (\textit{Ibid.}, 136). 
Dispositionalism should be well placed to avoid this criticism: Accounting for laws in terms of dispositions should ensure that they map directly onto the structures of reality as opposed to merely being an anthropocentric heuristic. \par



\subsection{Criticism of Dispositionalist Laws}

He targets some of his criticism specifically towards the dispositionalist account of laws: 
\begin{displayquote} Given the doubtless high number of natural kinds, and the doubtless even higher number of their characterising qualities, the theory would permit a vast number of laws of nature. (\textit{Ibid.}, 132.) \end{displayquote}
Any dispositionalist account of laws would contain every disposition of every natural kind. Mumford's criticism here is that we already have a huge number of supposed laws of nature - and bearing in mind that \textit{all} laws of nature would have to be articulated through the vehicle of natural kinds - we will end up committed to a hugely bloated account of laws of nature.\par
This criticism arguably does not directly engage with the crux of the issue. 
While a massively bloated account of the laws may not be desirable, any non-eliminativist theory of laws will end up with at least a cluttered ontology in attempting to capture every natural phenomenon. 
Mumford fails to show why the dispositionalist account of laws would be, in particular, less correct than others. 
Moreover, while dispositionalism leads to an untidy conclusion, this is only ever reason to state that it is \textit{undesirable} as opposed to incorrect.\par

Mumford produces two direct criticisms of dispositionalist laws: 
1.) Laws are not necessary if we accept causal powers, and; 2.) Any use of the term `laws' is misleading if the term just refers to causal powers.\par 

Firstly, laws are not necessary if we have causal powers. As established at the start of this article, having a governing role is an essential feature of a law. 
Yet, it is unclear how laws necessarily govern causal powers: Holding laws to govern dispostions externally leads to quidditism (as they are not necessarily manifested) and so laws can only govern internally (\textit{Ibid.}, 158). 
`Quidditism' is the view that properties manifest themselves only contingently. It is not compatible with dispositionalism, as causal powers must necessarily manifest themselves (even if they are never activated). 
If laws can only govern internally, then they do not provide any additional purpose and so are unnecessary. Those who subscribe to a dispositionalist view should feel comfortable with the transferal of necessity from laws to causal powers. 
While this may not guarantee the traditional logical necessity associated with the laws of nature, Mumford suggests that ``natural necessity'' should suffice (\textit{Ibid.}, 181).

Secondly, the use of the term `laws' is misleading if we assent to dispositionalism. 
Our understanding of causal powers is clear and more robust than our understanding of the laws of nature. Even taking `laws' to be purely metaphorical only serves to muddy the waters of our understanding due to the term's semantic inconsistency (\textit{Ibid.}, 200-2). 
Assuming that Mumford's previous argument holds water, this is an important corollary to note. Philosophers of science should be hesitant to subscribe to unclear ontologies.

\section{Responses to Mumford}

Bird also subscribes to a dispositionalist metaphysic, but is keen to maintain a notion of laws in our ontology. 
He explains Mumford's eliminativism as being motivated by similar eliminativist arguments in science, citing the example of phlogiston; a substance once infamously postulated to explain combustion. 
It is at this point, however, that Bird believes the comparison ends. He claims that the comparison fails if one were to treat phlogiston and oxygen as ``one and the same'' - as he believes causal powers and the laws of nature to be. If the terms refer to the same entity/entities, then neither can be eliminated (Bird, 2007, 189). 
\begin{displayquote} Thus for Mumford's argument to eliminate laws, he needs to show that laws would have some character that prevents them from being identical to potencies, since that requires laws to govern what is in effect internal to them. (\textit{Ibid.}) \end{displayquote}
Bird is correct that the two terms can co-refer, although this raises the question of what the scientific protocol is for coping with co-referring expressions. Historically scientists have been careful to only use one term as a matter of convention. 
Often scientists who discover new phenomena are able to designate exactly what their discovery ought to be called in a scientific context. 
Notably in the 1990s, elements 103-109 were discovered during a naming controversy which required the International Union of Pure and Applied Chemistry to step in and solve the dispute. 
As Mumford has argued, the semantics of the term `law' are far from clear while the semantics of casual powers are much better understood. If both terms co-refer, it would be only scientific to adopt the term with the clearest definition.\par
While Bird is also correct in his assertion that no governance is required in order for the laws to supervene (\textit{Ibid.}, 194), removing the criterion of governance from our notion of laws is not straightforward.  
Maintaining that laws have an explanatory role \textit{without} a governing role is just reiterating that the laws of nature are causal powers. This may cause problems with overdetermination. 
Overdetermination is a persistent problem for metaphysicians; the problem occurs where too many causes are cited for instantiating a specific phenomenon. 
If two men are pushing a mine-cart then an observer may wonder if the cart's movement is caused by two forces of men pushing, or the total force of their pushing. 
If two forces are taken into account but the observer fails to divide the total force, they may mistakenly believe that the cart should move twice as fast. 
Bird briefly addresses these worries, claiming that, ``Laws can explain by virtue of their being themselves explained by potencies.'' (\textit{Ibid.}, 197.) 
He is saying that the burden of explanation is entirely absorbed by the causal powers - but this is not wholly convincing. 
The two entities share exactly the same properties by co-referring, but if one asserts that they are ontologically distinct then there can be ambiguity over whether they both play the exact same causal role. 
In this case features of the natural world could be seen to be doubly instantiated by both the laws of nature and by causal powers. \par 
Treating causal powers and the laws of nature as distinct entities clutter our ontology and muddy our understanding of causation.

\section{Summary}

Overall, Mumford makes a convincing case. Although it is worth once again reiterating that his most damaging arguments presuppose dispositionalism. 
Bird's defence of laws suggests that we should be careful not to overstate the eliminativist case but when one considers the practicalities of using co-referring terms in science his defence is unconvincing. 
Use of the term `laws' does not appear to be compatible with scientific method. There are also metaphysical concerns regarding overdetermination that Bird must address further. \par 
Mumford's argument does not allow us to fully dispense with laws as informal explanatory devices yet he does provide good reason to reconsider their place in our ontology. 

\section{Bibliography}

- Armstong, D.M.: \textit{A World of States of Affairs}, Cambridge University Press, 1997.
\par \noindent \newline
- Bird, A: \textit{Nature's Metaphysics: Laws in Nature}, Oxford University Press, 2007.
\par \noindent \newline
- Ellis, B.: \textit{The Philosophy of Nature}, Acumen Publishing, 2002.
\par \noindent \newline
- Mumford, S.: \textit{Laws in Nature}, Routledge, 2004.
\par \noindent \newline
- Tugby, M: ``Platonic Dispositionalism", \textit{Mind} Vol. 122 486, 451-480, 2013.


\end{document}
